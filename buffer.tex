\section{Buffers, Files}


%%%%%%%%%%%%%%%%%%%%%%%%%%%%%%%%%%%%%%%%%%%%%%%%
\subsection*{File Exploration}

\href{https://vonheikemen.github.io/devlog/tools/using-netrw-vim-builtin-file-explorer/}{\texttt{Explore}}\textit{, a part of the native }\href{https://neovim.io/doc/user/pi\_netrw.html}{\texttt{netrw}}\textit{ package, is a useful utility for browsing the filesystem; invoke with }\texttt{:(L|S|R|N|H|T)ex}\textit{. Some commands \href{https://gist.github.com/danidiaz/37a69305e2ed3319bfff9631175c5d0f}{here}. Alternatively, use a plugin like }\href{https://github.com/nvim-tree/nvim-tree.lua}{\texttt{nvim.tree}}\textit{ or }\href{https://github.com/lambdalisue/fern.vim}{\texttt{fern.vim}}\textit{. For git exploration, use }\href{https://github.com/tpope/vim-fugitive}{\texttt{fugitive}}\textit{.}


%%%%%%%%%%%%%%%%%%%%%%%%%%%%%%%%%%%%%%%%%%%%%%%%
\subsection*{Buffers}

\entry{25mm}{:buffer}{list all}\\
\entry{25mm}{:[\#]bnext}{next buf.}\\
\entry{25mm}{:[\#]bprevious}{prev. buf.}\\
\entry{25mm}{:blast}{last buf}\\
\entry{25mm}{:brewind}{first buf}\\
\entry{25mm}{:set hidden}{hide all others}\\
\entry{25mm}{:sbnext}{split \& edit}\\
\entry{25mm}{:baddd <file>}{add indicated buffer}\\
\entry{25mm}{:bdelete}{discard buffer}\\
\entry{25mm}{:bunload}{remove buffer}\\


%%%%%%%%%%%%%%%%%%%%%%%%%%%%%%%%%%%%%%%%%%%%%%%%
\subsection*{Files}

\entry{25mm}{:ls}{ls open files}\\
\entry{25mm}{:e[dit] <file>}{open file}\\
\entry{25mm}{:enew}{open new file}\\
\entry{25mm}{:view <file>}{open read-only}\\
\entry{25mm}{:vi <file>}{same as :e}\\
\entry{25mm}{:next}{cycle to next}\\
\entry{25mm}{:previous}{cycle back}\\
\entry{25mm}{:wnext[!]}{write + next}\\
\entry{25mm}{:args}{list open files}\\
\entry{25mm}{:args <\#>}{edit \#th cmd-line file}\\
\entry{25mm}{CTRL-\textasciicircum}{goto last edited}\\


%%%%%%%%%%%%%%%%%%%%%%%%%%%%%%%%%%%%%%%%%%%%%%%%
\subsection*{Windows}
\entry{25mm}{:sp[lit]}{screen \underline{sp}lit}\\
\entry{25mm}{:splitbelow}{splitbelow}\\
\entry{25mm}{:vs[plit]}{\ul{v}ertical \ul{s}plit}\\
\entry{25mm}{:10vsplit}{split is 10 lines}\\
\entry{25mm}{:only}{keep only this }\\
\entry{25mm}{:hide}{close this w. }\\
\entry{25mm}{:[w|q]all[!]}{\ul{w}rite, \ul{q}uit all }\\
\entry{25mm}{:all <file>*}{open all }\\
\entry{25mm}{3<C-W>w}{goto specific (3\textsuperscript{rd}) }\\
\entry{25mm}{<C-W>\_}{maximize current }\\
\entry{25mm}{<C-W>w}{cycle windows}\\
\entry{32mm}{<C-W>j \quad <C-W>k}{up, down wind.}\\
\entry{32mm}{<C-W>$\uparrow$ <C-W>$\downarrow$}{up, down wind. }\\
\entry{32mm}{[\#]<C-W>+ <C-W>-}{expand, contract }\\
\entry{32mm}{[\#]<C-W>=\quad <C-W>\_}{sizes equal, exact }\\
\entry{32mm}{<C-r>\quad <C-R>}{\ul{r}otate win's on screen }\\
\entry{32mm}{<C-W>O}{quit all \underline{O}thers }\\


%%%%%%%%%%%%%%%%%%%%%%%%%%%%%%%%%%%%%%%%%%%%%%%%
\subsection*{Registers}

\textit{Unlike other editors, Vim:}\\
\begin{itemize}
    \item has multiple clipboards (\say{registers})
    \item can filter registers
    \item can cut \& paste on many files
\end{itemize}

\entry{25mm}{\textquotedbl ay3w}{yank 3w into reg \say{a} }\\
\entry{25mm}{\textquotedbl Byy}{append (\underline{B} vs b) to reg \say{b} }\\
\entry{25mm}{\textquotedbl <reg>[p|P]}{\underline{P}aste from <reg> }\\
\entry{25mm}{\textquotedbl*[y|P]}{\underline{y} to, \underline{P} from system clip. }\\
\entry{25mm}{yy}{yanks into \say{unnamed} reg }\\
\entry{25mm}{:registers}{show registers }\\

\textit{Special Registers:}\\
\begin{itemize}
    \item \begin{code}\textquotedbl@\end{code} ... 
    \item \begin{code}\textquotedbl"\end{code} ... \say{unnamed} (last yanked)
    \item \begin{code}\textquotedbl.\end{code} ... last inserted
    \item \begin{code}\textquotedbl\#\end{code} ... alternate file
    \item \begin{code}\textquotedbl0\end{code} ... last yanked
    \item \begin{code}\textquotedbl\%\end{code} ... file name
    \item \begin{code}\textquotedbl/\end{code} ... last search
    \item \begin{code}\textquotedbl:\end{code} ... last command
    \item \begin{code}\textquotedbl*\end{code} ... mouse clipboard
    \item \begin{code}\textquotedbl=\end{code} ... enter expression
    \item \begin{code}\textquotedbl\_\end{code} ... black hole
\end{itemize}
\ \\
