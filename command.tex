\section{Command Mode}

\subsection*{Common Commands}

\entry{34mm}{:w[rite][q]}{write [\&quite]}\\
\entry{34mm}{ZZ\quad ZQ}{ibid, alias}\\
\entry{34mm}{:help}{vim help}\\
\entry{34mm}{:help rust}{on a topic}\\
\entry{34mm}{u\quad <C-R>}{\ul{u}ndo / redo}\\
\entry{34mm}{<C-K><char><char>}{insert \href{http://vimdoc.sourceforge.net/htmldoc/digraph.html}{digraph}}\\
\entry{34mm}{<C-V>u<unicode>}{insert raw \href{https://www.serverwatch.com/guides/unicode-characters-in-vim/}{unicode}}\\
\entry{34mm}{:digraphs}{list digraphs}\\
\entry{34mm}{<C-G>}{file loc}\\
\entry{34mm}{:source <file>}{load a file}\\
\entry{34mm}{@:}{repeat last cmd}\\
\entry{34mm}{:!<cmd>}{run ext cmd}\\

\api
{2cm}{
ascii   \\
digraph \\
echo[m] \\
find    \\
fixdel 

}
{2cm}{
go      \\
gre     \\
ls      \\
mak     \\
normal  

}
{2cm}{
redraw!     \\
shell       \\
sleep       \\
statusline* \\
visual      

}
\stopapi

%%%%%%%%%%%%%%%%%%%%%%%%%%%%%%%%%%%%%%%%%%%%%%%
\subsection*{Statement Evaluation}
:execute \textquotedbl <str>\textquotedbl \\
... for example, to insert \say{;} at EOL \& return: \\
:execute "normal! mqA;\textbackslash<esc>`q"

%%%%%%%%%%%%%%%%%%%%%%%%%%%%%%%%%%%%%%%%%%%%%%%
\subsection*{Printing}

\entry{40mm}{:hardcopy > out.pdf}{to pdf }\\
\entry{40mm}{:set printfont=courier:h10}{font }\\


%%%%%%%%%%%%%%%%%%%%%%%%%%%%%%%%%%%%%%%%%%%%%%%
\subsection*{Abbreviations}

\entry{25mm}{:abbreviations}{show abbrevs }\\
\entry{25mm}{:ab <f> <t>}{set an abbrev. }\\

%%%%%%%%%%%%%%%%%%%%%%%%%%%%%%%%%%%%%%%%%%%%%%%
\subsection*{Syntax Highlighting}
:highlight <grp> <ttype>=<attr> \\
where <ttype> $\in$ term, cterm, gui \& attr $\in$: \\

\api
{2cm}{
bold    \\
{[\#rrggbb]}

}
{2cm}{
underline   \\
{[col\_name]}

}
{2cm}{
reverse     \\
standout

}
\stopapi

%%%%%%%%%%%%%%%%%%%%%%%%%%%%%%%%%%%%%%%%%%%%%%%
\subsection*{Autocommands}
Syntax:
\begin{enumerate}
    \item :autocmd <event>* <ft\_glob> <cmd>
    \item :autocmd FileType <ft\_glob> <cmd>
\end{enumerate}
Examples:\\
\begin{tabular}{l}

\begin{lstlisting}
:autocmd BufNewFile *.txt :write
\end{lstlisting} \\

\begin{lstlisting}
:autocmd BufWritePre,BufRead * ...
\end{lstlisting} \\

\end{tabular}

Best to use \say{augroups}:
\begin{lstlisting}
:augroup <grp_name>
:    autocmd! ...
:augroup END
\end{lstlisting}

%%%%%%%%%%%%%%%%%%%%%%%%%%%%%%%%%%%%%%%%%%%
\subsection*{Filters}

\entry{25mm}{!15G<cmd>}{filter, here to line 15 }\\
\entry{25mm}{!15G sort}{example filter }\\
\entry{25mm}{!!<cmd>}{single-line filter }\\
\entry{25mm}{!!date}{idiomatic: insert date }\\
\entry{25mm}{!!ls}{idiomatic: f.s. contents }\\


%%%%%%%%%%%%%%%%%%%%%%%%%%%%%%%%%%%%%%%%%%%
\subsection*{Built-in Functions}

\begin{multicols}{3}
\scriptsize
\begin{itemize}[label={}]
    \item append
    \item arg[c|v]
    \item browse
    \item bufexists
    \item bufloaded
    \item bufname
    \item buf[win]nr
    \item byte2ln
    \item char2nr
    \item col
    \item confirm
    \item delete
    %\item did\_filetype
    \item escape
    \item exists
    \item expand
    \item filereadable
    \item fnamemod.
    \item getcwd
    \item getftime
    \item getline
    \item getwinposx
    \item glob[path]
    \item has
    \item histadd
    \item histdel
    \item histget
    \item histnr
    \item hlexists
    \item hlID
    \item hostname
    \item input
    \item isdirectory
    \item libcal
    \item line
    \item line2byte
    \item localtime
    \item maparg
    \item mapcheck
    \item match
    \item matchend
    \item matchstr
    \item nr2char
    \item rename
    \item setline
    \item shellescape
    \item strftime
    \item strlen
    \item strpart
    \item strtrans
    \item substitute
    \item synID
    \item synIDattr
    \item system
    \item tempname
    \item visualmode
    \item virtcol
    \item winbufnr
    \item winheight
    \item winnr
\end{itemize}

\end{multicols}


%%%%%%%%%%%%%%%%%%%%%%%%%%%%%%%%%%%%%%%%%%%
\subsection*{Sessions}

\entry{25mm}{:mksession <file>}{save to file }\\
\entry{25mm}{:source <file>}{restore session }\\
\textit{Setting :set sessionoptions=<opt>* where <opt> $\in$ \{buffers, globals, winpos, winsize, resize, etc\}, allows you to selectively save parA}ts of a session.

\ \\